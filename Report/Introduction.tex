\section{Introduction}

Optimization is a problem-solving method which aims to find the most advantageous parameters for a given model. The model is known to the optimizer and accepts inputs while producing outputs. Usually the problem can be formulated in such a way that we seek to minimize the output value of the model or the output of some function which transforms the models output into a fitness score. Because of this the process is often referred to as minimization. It becomes obvious that this is useful when considering optimizing the layout of a circuit in order to minimize the power consumption. To achieve this the optimizer looks for combinations of parameters which let the model produce the best output to a given input~\cite{Eiben2015_origins}.


When dealing with simple mathematical models, optimization can be achieved using analytical methods, often calculating the derivative of the functional model, but these methods prove difficult to adapt to complex models which exhibit noisy behavior. Additionally, the analytical model is not always known, which makes it impossible to use such methods~\cite{xiong2015walk}. The field of evolutionary computation (EC), a subset of computational intelligence (CI), which is further a subset of artificial intelligence (AI), contains algorithms which are well suited to solving these kinds of optimization problems~\cite{Michalewicz1997,zhang2015comprehensive}.

Evolutionary computation focuses on problem solving algorithms which draw inspiration from natural processes. It is closely related to the neighboring field of swarm intelligence (SI), which often is, and in this thesis will be, included as a subset of EC. The basic rationale of the field is to adapt the mathematical models of biological Darwinian evolution to optimization problems. The usefulness of this can be illustrated by imagining that an organism acts as an ``input'' to the ``model'' of it's natural environment and produces an ``output'' in the form of offspring. Through multiple iterations biological evolution culls the population of organisms, only keeping the fit specimen, to produce organisms which become continuously better adapted to their environments. Evolutionary computation is, however, not merely confined to Darwinian evolution, but also includes a multitude of methods which draw from other natural processes such as cultural evolution and animal behavior~\cite{engelbrecht2007computational}.

The purpose of this thesis is to explore the performance and usefulness of three emerging evolutionary algorithms: differential evolution, particle swarm optimization and estimation of distribution algorithms. These algorithms are newer inventions inspired by genetic algorithms, evolution strategy, evolutionary programming and genetic programming. Differential evolution is a straight-forward evolutionary algorithms which uses simple mathematic vector calculations to achieve it's results, while particle swarm optimization tries to mirror the behavior of swarming animals and estimation of distribution uses probability distributions to guess it's way forward. The intention is to test and compare these against each other on a set of standard mathematical benchmark functions and practical problems in machine learning involving neural networks and game controllers and then, if possible, develop a new or modified algorithm which improves upon the aforementioned ones in some aspect.

Research has been conducted on improving various evolutionary algorithms by hybridizing and extending them, which has resulted in a wide array of algorithms for both specific and general purposes. Since these algorithms accepts parameters which modify their efficiency, studies have been carried out which compare different combinations of parameters. My thesis will draw upon this work by comparing three algorithms both generally and on a very specific problem.

The results from the benchmark show which algorithm is best suited to handle various machine learning problems and presents the advantages of using the new algorithm. The new algorithm called DEDA (Differential Estimation of Distribution Algorithms) has shown promising results at both machine learning and mathematical optimization tasks.
