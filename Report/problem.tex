\section{Problem Formulation}

The purpose of this thesis is to explore the performance and usefulness of three emerging evolutionary algorithms: differential evolution (DE), particle swarm optimization (PSO) and estimation of distribution algorithms (EDA). The intention is to compare these against each other on a set of benchmark functions and practical problems in machine learning and then, if possible, develop a new or modified algorithm which improves upon the aforementioned ones in some aspect.

\paragraph{Research Questions}

The questions asked in the thesis are:
\begin{itemize}
  \item How do DE, PSO and EDA perform comparatively when applied to mathematical optimization problems?
  \item How do DE, PSO and EDA perform comparatively when applied to machine learning problems such as neural network optimization and artificial intelligence in games?
  \item How are DE, PSO and EDA suited to these different problems?
  \item Can an improved algorithm which draws inspiration from the design of DE, PSO and/or EDA outperform any of them in some/all of the aforementioned benchmarks and problem sets?
\end{itemize}

\paragraph{Motivation}

This research is interesting because it yields insight into the applicability of emerging evolutionary algorithms to currently popular machine learning methods such as artificial neural networks and also compares them more generally on generic mathematical optimization problems. The possibility of an improved novel algorithm which is better at handling machine learning problems also makes the work more interesting.

\paragraph{Outcomes}

The goals in this work are:
\begin{itemize}
  \item Benchmark DE, PSO, EDA on mathematical optimization problems
  \item Benchmark DE, PSO, EDA on machine learning problems
  \item Develop a new algorithm inspired by DE, PSO and/or EDA
  \item Benchmark the new algorithm
  \item Compare the new algorithm with DE, PSO and EDA and draw conclusions from the results
\end{itemize}

\paragraph{Limitations}

The scope of this work limits the number of algorithms which can be included in the testing. The individual algorithms also have numerous variations and parameters which can dramatically affect their behavior and it will not be possible to take all these variations into considerations. Furthermore, the benchmarking will be restricted to a standard set of testing functions which may or may not provide reliable information regarding the general usability of the algorithms. Since evolutionary algorithms have a large number of potential and actual use-cases, the practical testing will only concern a small subset of these and may therefore not provide accurate data for all possible use-cases.
